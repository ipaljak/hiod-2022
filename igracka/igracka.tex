%%%%%%%%%%%%%%%%%%%%%%%%%%%%%%%%%%%%%%%%%%%%%%%%%%%%%%%%%%%%%%%%%%%%%%
% Problem statement
\begin{statement}[
  problempoints=100,
  timelimit=1 sekunda,
  memorylimit=512 MiB,
]{Igračka}

Malom Bernardu dosadila je Rubikova kocka te se odlučio zabaviti novom
igračkom. Igračku možemo zamisliti kao riječ $S$ koja se sastoji od slova $a$,
$b$ i $c$.

U jednom potezu, Bernard može odabrati bilo koja tri uzastopna slova te im
obrnuti redoslijed. Primjerice, ako je trenutno stanje igračke $aabc$,
Bernard može obrtanjem prvih triju slova postići riječ $baac$, odnosno
obrtanjem posljednjih triju slova postići riječ $acba$.

Igračka je valjana ako je nekim redoslijedom poteza moguće postići da se sva
pojavljivanja slova $a$ nalaze prije pojavljivanja svih slova $b$ i sva
pojavljivanja slova $c$ nalaze poslije svih pojavljivanja slova $b$ i slova
$a$. Pomozite Bernardu odrediti je li njegova igračka valjana ili strgana.

%%%%%%%%%%%%%%%%%%%%%%%%%%%%%%%%%%%%%%%%%%%%%%%%%%%%%%%%%%%%%%%%%%%%%%
% Input
\subsection*{Ulazni podaci}

U prvom je retku niz znakova koji predstavlja riječ $S$ iz teksta zadatka.

%%%%%%%%%%%%%%%%%%%%%%%%%%%%%%%%%%%%%%%%%%%%%%%%%%%%%%%%%%%%%%%%%%%%%%
% Output
\subsection*{Izlazni podaci}

U jedinom retku ispišite \texttt{"da"} ako je igračka valjana, odnosno \texttt{"ne"} ako nije.

%%%%%%%%%%%%%%%%%%%%%%%%%%%%%%%%%%%%%%%%%%%%%%%%%%%%%%%%%%%%%%%%%%%%%%
% Scoring
\subsection*{Bodovanje}

{\renewcommand{\arraystretch}{1.4}
  \setlength{\tabcolsep}{6pt}
  \begin{tabular}{ccl}
   Podzadatak & Broj bodova & Ograničenja \\ \midrule
    1 & 19 & $1 \le |S| \le 10$\\
    2 & 23 & $1 \le |S| \le 5\,000$ \\
    3 & 58 & $1 \le |S| \le 100\,000$ \\
\end{tabular}}

%%%%%%%%%%%%%%%%%%%%%%%%%%%%%%%%%%%%%%%%%%%%%%%%%%%%%%%%%%%%%%%%%%%%%%
% Examples
\subsection*{Probni primjeri}
\begin{tabularx}{\textwidth}{X'X}
\sampleinputs{test/igracka.dummy.in.1}{test/igracka.dummy.out.1} &
\sampleinputs{test/igracka.dummy.in.2}{test/igracka.dummy.out.2}
\end{tabularx}

<<<<<<< HEAD
<<<<<<< HEAD
\textbf{Pojašnjenje prvog probnog primjera:}
Moguće je postići željeni raspored obrnuvši raspored prvih triju znakova:
$\underline{baa}bcc \rightarrow aabbcc$
=======
\textbf{Pojašnjenje prvog probnog primjera:} Bernard može obrnuti poredak
prvih triju slova te tako dobiti riječ $aabbcc$, pa je njegova igračka
valjana.
>>>>>>> d2d5f440c18c767ba0e9dee0af60f7bfe8937a88
=======
\textbf{Pojašnjenje prvog probnog primjera:} Bernard može obrnuti poredak
prvih triju slova te tako dobiti riječ $aabbcc$, pa je njegova igračka
valjana.
>>>>>>> d2d5f440c18c767ba0e9dee0af60f7bfe8937a88

%%%%%%%%%%%%%%%%%%%%%%%%%%%%%%%%%%%%%%%%%%%%%%%%%%%%%%%%%%%%%%%%%%%%%%
% We're done
\end{statement}

%%% Local Variables:
%%% mode: latex
%%% mode: flyspell
%%% ispell-local-dictionary: "croatian"
%%% TeX-master: "../hio.tex"
%%% End:
