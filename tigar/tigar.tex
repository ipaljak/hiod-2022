%%%%%%%%%%%%%%%%%%%%%%%%%%%%%%%%%%%%%%%%%%%%%%%%%%%%%%%%%%%%%%%%%%%%%%
% Problem statement
\begin{statement}[
  problempoints=100,
  timelimit=1 sekunda,
  memorylimit=512 MiB,
]{Tigar}

Mladi Luka sudjelovao je na brojnim ljetnim kampovima mladih informatičara.
Kao i ostali polaznici, pohađao je predavanja, rješavao algoritamske zadatke,
jeo puding od čokolade, igrao picigin, \ldots

Međutim, nikada nije sudjelovao u društvenim igrama, već je samo pomno
promatrao ostale polaznike kako se zabavljaju. Jednom je prilikom tako
promatrao svoje prijateljice, Emu i Paulu, kako igraju društvenu igru
\textit{Set}. Cijelo su ga vrijeme djevojke nagovarale da se igra s njima,
no Luka se nije dao samo tako nagovoriti.

``Ako želite da se igram s vama, morat ćete mi dokazati da ste dostojne
  protivnice. Zaigrajte moju igru, \textit{igru mladog tigra}, na način da je
  svaki vaš potez optimalan. Ako to napravite, pridružit ću vam se i pobjedit ću
  vas u setu.''

\textit{Igra mladog tigra} odvija se na stablu koje će Luka izabrati, a ono se
sastoji od $N$ neobojenih čvorova numeriranih prirodnim brojevima od $1$ do
$N$. Igračice poteze vuku naizmjence, a igru će započeti Ema. Ema će u svom
prvom potezu odabrati neki čvor stabla i obojit će ga. U svakom će sljedećem
potezu igračica obojiti neki neobojeni čvor koji je susjedan čvoru koji je
bio obojen u prošlom potezu. Igra završava kad neka igračica ne može
napraviti potez. Emin cilj je da igra što dulje traje, odnosno da se odigra
najveći mogući broj poteza, dok je Paulin cilj da igra što kraće traje,
odnosno da se odigra najmanji mogući broj poteza.

Napišite program koji će odrediti broj poteza koji će se odigrati uz
pretpostavku da obje igračice igraju optimalno.

%%%%%%%%%%%%%%%%%%%%%%%%%%%%%%%%%%%%%%%%%%%%%%%%%%%%%%%%%%%%%%%%%%%%%%
% Input
\subsection*{Ulazni podaci}

U prvom je retku prirodan broj $N$ iz teksta zadatka.

U $i$-tom od idućih $N - 1$ redaka su prirodni brojevi $a_i$ i $b_i$
$(1 \le a_i, b_i \le N)$ koji označavaju da postoji brid između čvorova
s oznakama $a_i$ i $b_i$.

Bridovi će biti takvi da tvore stablo --- jednostavan, povezan graf koji ne
sadrži ciklus.

%%%%%%%%%%%%%%%%%%%%%%%%%%%%%%%%%%%%%%%%%%%%%%%%%%%%%%%%%%%%%%%%%%%%%%
% Output
\subsection*{Izlazni podaci}

U jedinom retku ispišite broj poteza koji će se odigrati ako igračice igraju
optimalno.

%%%%%%%%%%%%%%%%%%%%%%%%%%%%%%%%%%%%%%%%%%%%%%%%%%%%%%%%%%%%%%%%%%%%%%
% Scoring
\subsection*{Bodovanje}

{\renewcommand{\arraystretch}{1.4}
  \setlength{\tabcolsep}{6pt}
  \begin{tabular}{ccl}
   Podzadatak & Broj bodova & Ograničenja \\ \midrule
    1 & 5 & $2 \le N \le 100\,000$, svaki čvor $x=1,\ldots,N-1$ je povezan s čvorom $x+1$.\\
    2 & 52 & $2 \le N \le 5\,000$ \\
    3 & 43 & $2 \le N \le 100\,000$
\end{tabular}}

%%%%%%%%%%%%%%%%%%%%%%%%%%%%%%%%%%%%%%%%%%%%%%%%%%%%%%%%%%%%%%%%%%%%%%
% Examples
\subsection*{Probni primjeri}
\begin{tabularx}{\textwidth}{X'X'X}
\sampleinputs{test/tigar.dummy.in.1}{test/tigar.dummy.out.1} &
\sampleinputs{test/tigar.dummy.in.2}{test/tigar.dummy.out.2} &
\sampleinputs{test/tigar.dummy.in.3}{test/tigar.dummy.out.3}
\end{tabularx}

%%%%%%%%%%%%%%%%%%%%%%%%%%%%%%%%%%%%%%%%%%%%%%%%%%%%%%%%%%%%%%%%%%%%%%
% We're done
\end{statement}

%%% Local Variables:
%%% mode: latex
%%% mode: flyspell
%%% ispell-local-dictionary: "croatian"
%%% TeX-master: "../hio.tex"
%%% End:
