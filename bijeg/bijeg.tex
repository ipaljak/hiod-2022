%%%%%%%%%%%%%%%%%%%%%%%%%%%%%%%%%%%%%%%%%%%%%%%%%%%%%%%%%%%%%%%%%%%%%%
% Problem statement
\begin{statement}[
  problempoints=100,
  timelimit=1 sekunda,
  memorylimit=512 MiB,
]{Bijeg}

``\textit{Tu sam vam ja provel svoju mladost, od dvajst prve do dvajst i sedme
godine, tak da znam kak tu vetar puše}'', Robert Markovčić, 2018.

Robert je proveo sedam godina svoje mladosti u lepoglavskoj kaznionici koju je
popularno nazvao \textit{Hrvatski Alcatraz}. Kao i svi ostali zatvorenici, bio
je potpuno nevin i nepravedno osuđen, te je često razmišljao o bijegu iz zatvora.
Ipak, njegova vjera u hrvatsko pravosuđe, demokratske vrijednosti i bolje sutra
bila je jača od poriva za bijegom, te je svoju robiju uredno odslužio.

Nije prošlo ni dvadeset ljeta, a Robert nam se ponovno nalazi u lepoglavskoj
ćeliji. Naravno, kao i prethodnog puta, Robert je potpuno nevin i nepravedno
osuđen. Međutim, Robert više nije mlad. Robert više ne vjeruje u hrvatsko
pravosuđe.  Robert više ne vjeruje u demokratske vrijednosti. Robert više ne
vjeruje u bolje sutra. Robert je pobjegao iz zatvora!

Lokalni organi reda promptno su reagirali, a institucije su radile svoj posao.
Stoga su u najkraćem mogućem roku na karti sjeverozapadne hrvatske istaknuli
$N$ područja od interesa, te ih numerirali prirodnim brojevima od $1$ do $N$.
Područje s oznakom $1$ označava lepoglavsku kaznionicu, dok preostlih $N-1$
područja označava potencijalne lokacije gdje bi se moglo nalaziti Robertovo
sklonište. Također su na karti istaknuli i $M$ prohodnih, dvosmjernih staza
kojima se Robert mogao kretati između određenih parova područja. Za svaku
stazu poznata je i njena duljina u kilometrima.

Potom su detektivi zatražili pomoć viših stručnih savjetnika iz Ministarstva za
utilizaciju vidovnjaka u kriminalistici. Više stručno povjerenstvo je temeljitom
analizom karte s istaktnutim područjima i stazama došlo do sljedećih zaključaka:

\begin{itemize}
  \item Sklonište bjegunca Roberta jedno je od $K$ područja s oznakama
        $q_1$, $q_2$, \ldots, $q_K$.
  \item Za neke je staze utvrđeno da će ih Robert sigurno proći na putu do svog
        skloništa.
  \item Robert će do skloništa putovati (nekim) najkraćim putom, gdje duljinu
        puta definiramo kao sumu duljina svih staza kojima će Robert proći.
\end{itemize}

Temeljem ovog izvještaja detektivi su zaključili da bi sada bilo korisno
kontaktirati algoritamske stručnjake. Dakako, radi se o vama --- najboljim
mladim hrvatskim informatičarkama. Možete li na temelju karte i izvještaja višeg
stručnog povjerenstva dodatno suziti skup područja među kojima se nalazi
sklonište bjegunca Roberta?

%%%%%%%%%%%%%%%%%%%%%%%%%%%%%%%%%%%%%%%%%%%%%%%%%%%%%%%%%%%%%%%%%%%%%%
% Input
\subsection*{Ulazni podaci}

U prvom su retku brojevi $N$, $M$ $(M \le \frac{N(N-1)}{2})$ i $K$ $(K < N)$
iz teksta zadatka.

U $i$-tom od sljedećih $M$ redaka su brojevi $a_i$, $b_i$, $d_i$ i $x_i$ koji
predstavljaju stazu dugačku $d_i$ $(1 \le d_i \le 10^9)$ kilometara koja
direktno spaja područja s oznakama $a_i$ i $b_i$ $(1 \le a_i, b_i \le N, a_i
\ne b_i)$. Dodatno, broj $x_i$ jednak je $1$ ako je povjerenstvo utvrdilo da će
Robert tijekom bijega proći $i$-tom stazom, odnosno jednak je $0$ ako
povjerenstvo to nije utvrdilo. Za barem jednu stazu vrijedit će $x_i = 1$,
staze biti takve da postoji put između svakog para istaknutih područja, te se u
ulazu neće ponavljati.

U sljedećem je retku $K$ brojeva $q_1$, $q_2$, \ldots, $q_K$ $(1 < q_i \le N)$
iz teksta zadatka.

Testni podaci će biti takvi da su zaključci stručnog povjerensta konzistentni.
Odnosno, sve staze za koje je $x_i = 1$ nalazit će se na jednom od najkraćih
putova između zatvora (područja s oznakom $1$) i nekog od potencijalnih
Robertovih skloništa (područja $q_1$, $q_2$, \ldots ili $q_K$).

\clearpage

%%%%%%%%%%%%%%%%%%%%%%%%%%%%%%%%%%%%%%%%%%%%%%%%%%%%%%%%%%%%%%%%%%%%%%
% Output
\subsection*{Izlazni podaci}

U $i$-tom retku ispišite broj $1$ ako se temeljem zaključaka povjerenstva područje
$q_i$ i dalje može smatrati potencijalnim skloništem. Odnosno, ispišite $1$ ako
se sve staze za koje je $x_i = 1$ nalaze na jednom od najkraćih putova između
područja s oznakom $1$ i područja s oznakom $q_i$. U protivnom, u $i$-ti redak
ispišite broj $0$.

%%%%%%%%%%%%%%%%%%%%%%%%%%%%%%%%%%%%%%%%%%%%%%%%%%%%%%%%%%%%%%%%%%%%%%
% Scoring
\subsection*{Bodovanje}

U svim podzadacima vrijedi $2 \le N \le 100\,000$.

{\renewcommand{\arraystretch}{1.4}
  \setlength{\tabcolsep}{6pt}
  \begin{tabular}{ccl}
   Podzadatak & Broj bodova & Ograničenja \\ \midrule
    1 & 19 & $M = N - 1$, odnosno staze i područja čine stablo. \\
    2 & 21 & $K \le 100$, te za točno jednu stazu vrijedi $x_i = 1$. \\
    3 & 23 & $K \le 100$ \\
    4 & 37 & nema dodatnih ograničenja.
\end{tabular}}

%%%%%%%%%%%%%%%%%%%%%%%%%%%%%%%%%%%%%%%%%%%%%%%%%%%%%%%%%%%%%%%%%%%%%%
% Examples
\subsection*{Probni primjeri}
\begin{tabularx}{\textwidth}{X'X'X}
\sampleinputs{test/bijeg.dummy.in.1}{test/bijeg.dummy.out.1} &
\sampleinputs{test/bijeg.dummy.in.2}{test/bijeg.dummy.out.2} &
\sampleinputs{test/bijeg.dummy.in.3}{test/bijeg.dummy.out.3}
\end{tabularx}

%%%%%%%%%%%%%%%%%%%%%%%%%%%%%%%%%%%%%%%%%%%%%%%%%%%%%%%%%%%%%%%%%%%%%%
% We're done
\end{statement}

%%% Local Variables:
%%% mode: latex
%%% mode: flyspell
%%% ispell-local-dictionary: "croatian"
%%% TeX-master: "../hio.tex"
%%% End:
