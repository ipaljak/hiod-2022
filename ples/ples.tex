%%%%%%%%%%%%%%%%%%%%%%%%%%%%%%%%%%%%%%%%%%%%%%%%%%%%%%%%%%%%%%%%%%%%%%
% Problem statement
\begin{statement}[
  problempoints=100,
  timelimit=1 sekunda,
  memorylimit=512 MiB,
]{Ples}

Stjepan, rastrgan između raznih karijernih puteva, odlučio je \st{odj}odustati
od informatičke karijere i postati koreograf!

No, kako Stjepan ipak ne može zanemariti svoju matematičku prošlost, odlučio je
na svoju koreografiju primijeniti matematičke metode plesa, popularni predmet
MMP na njegovom fakultetu. Naime, njegova koreografija uključuje $K$ plesača
te se sastoji od $N$ koraka. Na početku svaki se plesač nalazi na nekoj od
$K$ pozicija označenih brojevima od $1$ do $K$. Korak možemo prikazati nizom
od $K$ različitih brojeva $(a_1, a_2, \ldots, a_K)$, gdje broj $a_i$ označava
da će plesač, koji se trenutno nalazi na poziciji $i$, graciozno
otplesati do pozicije $a_i$.

Dodatno, Stjepan neku koreografiju smatra \textit{lijepom} ako se plesači na
kraju koreografije nalaze na istim pozicijama kao i na početku. Stjepan
\textit{blistavost} koreografije definira kao broj uzastopnih podnizova koji
zasebno čine \textit{lijepu} koreografiju. Stjepan vas sada moli da
izračunate \textit{blistavost} njegove koreografije!

%%%%%%%%%%%%%%%%%%%%%%%%%%%%%%%%%%%%%%%%%%%%%%%%%%%%%%%%%%%%%%%%%%%%%%
% Input
\subsection*{Ulazni podaci}

U prvom su retku prirodni brojevi $N$ i $K$ iz teksta zadatka.

U $i$-tom od idućih $N$ redaka nalazi se $K$ brojeva, $a_1, a_2, \ldots,
a_k$ koji opisuju $i$-ti korak u plesu.

%%%%%%%%%%%%%%%%%%%%%%%%%%%%%%%%%%%%%%%%%%%%%%%%%%%%%%%%%%%%%%%%%%%%%%
% Output
\subsection*{Izlazni podaci}

U jedinom retku ispišite \textit{blistavost} koreografije.

%%%%%%%%%%%%%%%%%%%%%%%%%%%%%%%%%%%%%%%%%%%%%%%%%%%%%%%%%%%%%%%%%%%%%%
% Scoring
\subsection*{Bodovanje}

{\renewcommand{\arraystretch}{1.4}
  \setlength{\tabcolsep}{6pt}
  \begin{tabular}{ccl}
   Podzadatak & Broj bodova & Ograničenja \\ \midrule
    1 & 5 & $1 \le N \le 100\,000$, $K = 1$\\
    2 & 6 & $1 \le N \le 5\,000$, $K = 2$ \\
    3 & 11 & $1 \le N \le 100\,000$, $K = 2$ \\
    4 & 14 & $1 \le N \le 5\,000$, $1 \leq K \leq 5$ \\
    5 & 13 & $1 \le N \le 100\,000$, $1 \leq K \leq 5$ \\
    6 & 12 & $1 \le N \le 5\,000$, $1 \leq K \leq 20$ \\
    7 & 7 & $1 \le N \le 100\,000$, $1 \leq K \leq 20$ \\
    8 & 32 & $1 \leq NK \le 2\,000\,000$
\end{tabular}}

%%%%%%%%%%%%%%%%%%%%%%%%%%%%%%%%%%%%%%%%%%%%%%%%%%%%%%%%%%%%%%%%%%%%%%
% Examples
\subsection*{Probni primjeri}
\begin{tabularx}{\textwidth}{X'X}
\sampleinputs{test/ples.dummy.in.1}{test/ples.dummy.out.1} &
\sampleinputs{test/ples.dummy.in.2}{test/ples.dummy.out.2}
\end{tabularx}

\textbf{Pojašnjenje prvog probnog primjera:}
Lijepi uzastopni podnizovi su sljedeći: (1, 1), (5, 5), (3, 4),

 (3, 5)

%%%%%%%%%%%%%%%%%%%%%%%%%%%%%%%%%%%%%%%%%%%%%%%%%%%%%%%%%%%%%%%%%%%%%%
% We're done
\end{statement}

%%% Local Variables:
%%% mode: latex
%%% mode: flyspell
%%% ispell-local-dictionary: "croatian"
%%% TeX-master: "../hio.tex"
%%% End:
